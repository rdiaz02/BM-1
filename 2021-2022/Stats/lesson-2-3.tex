\synctex=1
% Created 2014-12-01 Mon 10:14
\documentclass[bigger]{beamer}
%% \usepackage[utf8]{inputenc}
\usepackage[latin1]{inputenc}
\usepackage{times} %% o.w. fonts look like shit
\usepackage[T1]{fontenc}
\usepackage{graphicx}
\usepackage{float}

\usepackage{fixltx2e}
\usepackage{longtable}
\usepackage{wrapfig}
\usepackage{rotating}
\usepackage[normalem]{ulem}
\usepackage{amsmath}
\usepackage{textcomp}
\usepackage{marvosym}
%% \usepackage{wasysym}
\usepackage{amssymb}
\usepackage{hyperref}
\usepackage{url}


\usepackage[begintext=\textquotedblleft,endtext=\textquotedblright]{quoting}

\tolerance=1000
\usepackage{CSIC2}
\setbeamertemplate{navigation symbols}{}
\usetheme{default}

\newcommand{\blu}[1]{{\textcolor {blue} {#1}}}
\newcommand{\green}[1]{{\textcolor {green} {#1}}}


\newcommand{\Burl}[1]{\blu{\url{#1}}}


\author{Ramon Diaz-Uriarte}
% \date{}
\date{2021-09-24}
\title{Lesson 2.3}
\hypersetup{
  pdfkeywords={},
  pdfsubject={},
  pdfcreator={Emacs 24.4.1 (Org mode 8.2.10)}}
\begin{document}

% \maketitle
% \begin{frame}{Outline}
% \tableofcontents
% \end{frame}



\begin{frame}
  \frametitle{Today}
  \begin{itemize}
  \item Paired data: how and why. An intuitive explanation
  \item Paired data: how it controls for subject differences. (A minimal formula)
  \item Plots for paired data
  \item Non-parametric procedures
  \item Non independent data
  \item Bayesian inference
  \item Confidence intervals are not always symmetric!
  \item Confidence interval and p-values: deriving the expression
  \item Power demo with R commander
  \item Central limit theorem demo with R commander
  \end{itemize}
\end{frame}

\end{document}

