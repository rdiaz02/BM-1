\synctex=1
% Created 2014-12-01 Mon 10:14
\documentclass[bigger]{beamer}
%% \usepackage[utf8]{inputenc}
\usepackage[latin1]{inputenc}
\usepackage{times} %% o.w. fonts look like shit
\usepackage[T1]{fontenc}
\usepackage{graphicx}
\usepackage{float}

\usepackage{fixltx2e}
\usepackage{longtable}
\usepackage{wrapfig}
\usepackage{rotating}
\usepackage[normalem]{ulem}
\usepackage{amsmath}
\usepackage{textcomp}
\usepackage{marvosym}
%% \usepackage{wasysym}
\usepackage{amssymb}
\usepackage{hyperref}
\usepackage{url}


\usepackage[begintext=\textquotedblleft,endtext=\textquotedblright]{quoting}

\tolerance=1000
\usepackage{CSIC2}
\setbeamertemplate{navigation symbols}{}
\usetheme{default}

\newcommand{\blu}[1]{{\textcolor {blue} {#1}}}
\newcommand{\green}[1]{{\textcolor {green} {#1}}}


\newcommand{\Burl}[1]{\blu{\url{#1}}}


\author{Ramon Diaz-Uriarte}
\date{2021-09-27}
%% \date{2021-09-24}
\title{Anova theory even simpler}
\subtitle{The 2 and a half pages from Peter Dalgaard, even more simplified.}
\hypersetup{
  pdfkeywords={},
  pdfsubject={},
  pdfcreator={Emacs 24.4.1 (Org mode 8.2.10)}}
\begin{document}

 \maketitle

% \begin{frame}
%   \frametitle{Anova theory even simpler}

%   The 2 and a half pages from Peter Dalgaard, even more simplified.
  
% \end{frame}

 % \begin{frame}{Outline}
% \tableofcontents
% \end{frame}



\begin{frame}
  \frametitle{What the $MS$ estimate}
  \begin{itemize}
  \item $MS_W$: an estimate of the variance, $\sigma^2$.
    \begin{itemize}
    \item {\tiny (The variance of what? Of the observations. Recall the
        first page, that says the error terms are assumed to come from a normal
        distribution of mean 0 and variance $\sigma^2$).}
    \end{itemize}

    \vspace*{20pt}
  \item $MS_B$:
    \begin{itemize}
    \item If the null hypothesis is true (i.e., all group means are equal, i.e.,
      $\mu_1 = \mu_2 = \mu_3 = \ldots$): $MD_B$ is also an estimate of the variance,
      $\sigma^2$.
          \vspace*{10pt}
    \item If not all means are the same (i.e., at least one true mean is
      different from the rest):  $MD_B$ is ``an estimate of the
      variance, $\sigma^2$ plus something else''
      \begin{itemize}
      \item That ``something else'' is the contribution that the differences
        between means make to the $SSD_B$. 
      \end{itemize}
    \end{itemize}
  \end{itemize}
\end{frame}

\begin{frame}
  \frametitle{What we expect the $MS$ ratio to be}
  Thus:
    \begin{itemize}
    \item If the null hypothesis is true, both $MS_B$ and $MS_W$ are estimating
      the same thing. Therefore, the ratio $\frac{MS_B}{MS_W}$ should be about 1.
    \item If the null hypothesis is not true (i.e., not all means are equal),
      $MS_B$ should be larger than $MS_W$. So the ratio $\frac{MS_B}{MS_W}$
      should be larger than 1.
    \end{itemize}

    The F-statistic is that ratio  $\frac{MS_B}{MS_W}$. We will compare it
    against the F distribution (the distribution of that ratio under the null).


    \vspace*{10pt}

    {\scriptsize The F distribution
    has, as parameters, the degrees of freedom of the MS in the numerator and the
    degrees of freedom of the MS in the denominator.}
\end{frame}



\begin{frame}
  \frametitle{What do SSD and MS stand for}
  \begin{itemize}
  \item The ``SSD'' stands for ``sum of squared deviations''. It is often written
    (as in much computer output) as ``Sum Sq'' or ``sums of squares'' or ``SS''.
  \item ``MS'': mean squares.
  \end{itemize}
\end{frame}


\end{document}

