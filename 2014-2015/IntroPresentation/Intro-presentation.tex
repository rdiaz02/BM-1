\documentclass[xcolor=dvipsnames]{beamer}
\mode<presentation>

\setbeamertemplate{itemize item}[ball]
%\usetheme[secheader]{Boadilla}
%\usetheme{boxes}
\usetheme{myBoadilla}
%\usetheme[secheader]{myBoadilla}
%\usetheme[left]{Goettingen}

%\usepackage{CSIC2}
\setbeamertemplate{itemize item}[ball]
 \setbeamertemplate{navigation symbols}{
      \insertslidenavigationsymbol
%%      \insertdocnavigationsymbol 
%%    {\footnotesize \insertframenumber} %% FIXME: delete if we use CSIC2
 }

\usepackage[absolute,overlay]{textpos}
\usepackage[english]{babel}
\usepackage[latin1]{inputenc}
\usepackage{times}
\usepackage[T1]{fontenc}
% Or whatever. Note that the encoding and the font should match. If T1
% does not look nice, try deleting the line with the fontenc.
%%\usepackage{fancybox}
\usepackage[lined]{algorithm2e}

\usepackage{url}
\newcommand{\cyan}[1]{{\textcolor {cyan} {#1}}}
\newcommand{\blu}[1]{{\textcolor {blue} {#1}}}
\newcommand{\Burl}[1]{\blu{\url{#1}}}
\newcommand{\red}[1]{{\textcolor {red} {#1}}}
\newcommand{\green}[1]{{\textcolor {green} {#1}}}
\newcommand{\mg}[1]{{\textcolor {magenta} {#1}}}
\newcommand{\og}[1]{{\textcolor {PineGreen} {#1}}}
\newcommand{\code}[1]{\texttt{\slshape\footnotesize #1}}
\newcommand{\myverb}[1]{{\footnotesize\texttt {\textbf{#1}}}}
\newcommand{\Rnl}{\ +\qquad\ }
\newcommand{\Emph}[1]{\emph{\mg{#1}}}
\newcommand{\gry}[1]{{\textcolor {gray} {#1}}}

\newcommand{\hanging}[1]{ 
\renewcommand{\baselinestretch}{0.7}\small\normalsize
\setlength{\parskip}{1.5ex plus0ex minus0ex}
  \noindent\hangindent=24pt
  \hangafter=1 
   {\scriptsize #1}
  %\hangindent\parindent
}   


\newcommand{\hango}[1]{ 
\renewcommand{\baselinestretch}{0.7}\small\normalsize
\setlength{\parskip}{1.5ex plus0ex minus0ex}
  \noindent\hangindent=24pt
  \hangafter=1 
   {\tiny #1}
  %\hangindent\parindent
}   


\newcommand{\hanginp}[2]{ 
  \renewcommand{\baselinestretch}{0.7}\small\normalsize
  \setlength{\parskip}{1.5ex plus0ex minus0ex}
  \noindent\hangindent=24pt
  \hangafter=1 
  {\scriptsize #1}{\hspace{2pt}\Large\red{#2}}
  % \hangindent\parindent
}   


%% use for itemize wihtin description
% \newenvironment{myitemize} {
%   \setlength{\leftmarginii}{-10mm}
%   \itemize
% }

% \newenvironment{myitemize}
% {\setlength{\leftmarginii}{-10mm}
% \begin{itemize}}
% {\end{itemize}}


\newenvironment{myitemize}
{\begin{itemize}}{\end{itemize}}


\newcommand{\mydes}[2][]{
  \vspace*{-1pt}
%%  \renewcommand{\baselinestretch}{0.7}\small\normalsize
  \begin{description}\item[{#2}]{#1}\end{description}
\vspace*{-8pt}
}


\usepackage{gitinfo}


\title{BM-1: Applied statistics}



\author[R. Diaz-Uriarte]{Ram�n Diaz-Uriarte} %% \\ \Burl{http://ligarto.org/rdiaz}}

\institute[]{Department of Biochemistry\\
Universidad Aut�noma de Madrid\\
Madrid, Spain\\
\texttt{ramon.diaz@iib.uam.es}\\
\Burl{http://ligarto.org/rdiaz}
}


% \date[22-09-2014]{22-September-2014}
% \date{\gitAuthorDate\ {\footnotesize (Rev: \gitAbbrevHash)}}
\date[22-09-2014]{22-September-2014\\ {\scriptsize (Rev: \gitAbbrevHash)}}


\begin{document}


\begin{frame}
  \titlepage
\end{frame}


% \begin{frame}
%    \frametitle{Outline}
%  {\small
%  \tableofcontents[subsectionstyle=hide]
%  }
%  \end{frame}




\begin{frame}
  \frametitle{Objectives}
  \begin{itemize}
  \item Be able to conduct, and interpret correctly, some basic types of
    statistical analysis of wide use in molecular biology et al.
  \item Develop your statistical intuition.
  \end{itemize}
\end{frame}


\begin{frame}
  \frametitle{Syllabus}
  \begin{itemize}
  \item Intro to R Commander
  \item Comparing two groups (and a single one)
  \item Comparing more than two groups: ANOVAs and linear models
  \item Basic categorical data analysis
  \item The multiple testing problem

% \vspace*{0.5cm}
% \item All sessions: computer work
  \end{itemize}
\end{frame}



\begin{frame}
  \frametitle{Why those topics}
  \begin{itemize}
  \item Indispensable
  \end{itemize}
\end{frame}



\begin{frame}
  \frametitle{Managing expectations}
  Is this all the stats knowledge I need for \ldots

\vspace*{15pt}
  \begin{itemize}
  \item my ``trabajo fin de master''? Most likely not.
  \item doing research? \textbf{Definitely not}.
  \end{itemize}
\end{frame}



\begin{frame}
  \frametitle{\ldots eh????!!!!}
  \begin{itemize}
  \item We cannot cover all you might need
  \item Develop statistical intuition: go learn on your own
  \item Know when to consult a statistician ({\footnotesize hint: as soon
      as possible})
  \end{itemize}
\vspace*{10pt}

What are we leaving out? Almost everything.


\vspace*{10pt}


{\footnotesize
(Remember analogy of first aid course vs.\ becoming a neurosurgeon)
}

\end{frame}





\begin{frame}
  \frametitle{Why R?}
  \begin{itemize}
  \item Free (as in ``free beer'' and as in ``free speech''): free software
  \item Available for all major operating systems
  \item High quality
  \item Huge library of contributed packages with added
    functionality (including BioConductor); increasing weekly
  \item The most used statistical system in bioinformatics and
    ``omics'' research
  \item Available in popular ``scientific workflows''
    (\Burl{http://en.wikipedia.org/wiki/Scientific_workflow_system}) such
    as Kepler, Taverna, Galaxy or Knime.
  \item \ldots

  \end{itemize}
\end{frame}







\begin{frame}
  \frametitle{How? R Commander}
  \begin{itemize}
  \item We will use R Commander (\Burl{http://socserv.mcmaster.ca/jfox/Misc/Rcmdr/})
  \item GUI + full transcript + markdown
  \item Also RStudio (secondary)
  \end{itemize}
\end{frame}


\begin{frame}
  \frametitle{How does it look?}
\end{frame}

%%% Say: I do not use for real, ever.


\begin{frame}
  \frametitle{(Don't tell anyone)}
  Our hidden agenda
  \begin{itemize}
  \item That you finish the class with a burning desire (to learn) to use
    R directly typing R commands (i.e., no mouse clicks).
  \end{itemize}
\end{frame}



\begin{frame}
  \frametitle{What do I need?}
  \begin{itemize}
  \item Computers in class will have it installed
  \item You MUST install it on your own computer
    \begin{itemize}
    \item if you bring your computer to class (highly recommended)
    \item for out-of-class work
    \item for the rest of your life
    \end{itemize}
  \end{itemize}


  \textbf{STOP! Some packages available in July and August 2014 are now
    deprecated. On 19-September-2014 I will provide updated notes that
    will work as of that day}.

\textbf{Thus, the text below is mostly, but not fully, correct as of now.}

  See email with instructions (sent on 2014-08-12) or link
  \Burl{http://ligarto.org/rdiaz/BM1-software-applied-stats.txt}
\vspace*{10pt}

or


\vspace*{10pt}
\Burl{https://github.com/rdiaz02/BM-1/blob/master/2014-2015/BM1-software-applied-stats.txt}
\end{frame}



% \begin{frame}
%   \frametitle{Two more packages: RcmdrPlugin.doBy and ISwR}
% We need to add a couple of things.
%   \begin{itemize}
%   \item You also will want to install the package ``ISwR'': I use one data
%     set for a couple of examples.
%     %% FIXME!!!
% \item Download it from \red{zz FIXME}:
%   \begin{itemize}
%   \item Linux and Mac:
%   \item Windows:
%   \end{itemize}

% \item You can try installing it. We will also explain it in class.
% \item But you MUST have downloaded it.
%   \end{itemize}
% \end{frame}

\begin{frame}
  \frametitle{One more package: RcmdrPlugin.doBy}

  \begin{itemize}
  \item You also need to install a modified version of the package ``RcmdrPlugin.doBy''.
    %% FIXME!!!
\item Download it from Moodle (file ``RcmdrPlugin.doBy'')
  
\item You can try installing it. We will also explain it in class.
  \begin{itemize}
  \item In RStudio go to ``Tools'', ``Install Packages''.
  \item In the entry ``Install from'', select ``Package Archive File''
    \item In the second entry, ``Package archive'', select the file you
      got (RcmdrPlugin.doBy\_0.1-4.tar.gz)
    \item Install
    \end{itemize}
\item But you MUST have downloaded it.
  \end{itemize}
\end{frame}

\begin{frame}
  \frametitle{Software in your laptops}
  \textbf{\red{Please make sure it works BEFORE coming to class}}
\end{frame}

\begin{frame}
  \frametitle{Can I use \ldots}
  \begin{itemize}
  \item SAS
  \item SPSS
  \item Minitab
  \item Matlab
  \item Stata
  \item Statgraphics
  \item Excel (don't!)
  \item \ldots
  \end{itemize}
  
  yes, of course!! \ldots


\pause 
\vspace*{10pt}

\ldots but you are on your own: we will not provide any help for systems
  other than R.

\end{frame}





\begin{frame}
  \frametitle{Class mechanics}
  \begin{itemize}
  \item A mixture of theory and practice
  \item Introduce a problem, work through it
  \item Much, much better if you work through the notes before class
  \item Yes, there will be partial repetition between
    classes/sections/teachers
  \end{itemize}
\end{frame}




\begin{frame}
  \frametitle{Class mechanics}
  \begin{itemize}
  \item Do NOT expect too much hand holding in terms of the ``where is the
    pull down menu for importing a file'', etc
  \item We (teachers) will:
    \begin{itemize}
    \item Explain concepts that you do not understand
    \item Help you get going
    \item Try to save you from despair
    \end{itemize}
  \item You do the learning work: explore and learn by
    doing/exploring. \textbf{Active learning}
  \end{itemize}
\end{frame}



\begin{frame}
  \frametitle{Working through the notes before class}
  Why?
\vspace*{15pt}
Because you will learn the material a lot better (\textbf{Active learning})
\begin{itemize}
\item try it on your own
\item find out what you don't know/understand (or makes no sense in the notes)
\item get it answered
\end{itemize}
\end{frame}


\begin{frame}
  \frametitle{Working through the notes before class}
  Why?
\vspace*{15pt}
Because there is a lot of material to cover (really: A LOT)

\begin{itemize}
\item No time to cover every single thing in the notes
\item If trivial stuff out of the way, we can deal with more interesting stuff
\end{itemize}

\end{frame}



\begin{frame}
  \frametitle{Working through the notes before class}
  Can you really do it? \textbf{YES!}

\vspace*{15pt}

Because all of you should have taken at least one statistics class.

\vspace*{15pt}


\end{frame}


\begin{frame}
  \frametitle{Working through the notes before class}

Thus \ldots

\begin{itemize}
\item This part of BM-1 should be a quick review
\item Focused on things you might have forgotten or tricky things you
  might not know
\item And with a strong practical component of ``teach me how to do it
  with available software, interpret it, and produce reports and graphics
  I can use in my TFM (and life at large)''.
\end{itemize}

\end{frame}





\begin{frame}
  \frametitle{Working through the notes before class}
  How?
\vspace*{15pt}
Working in groups a good idea
\begin{itemize}
\item Your peers are likely to understand better what is it you do not
  understand
\end{itemize}

\end{frame}




% \begin{frame}
%   \frametitle{Working through the notes before class}
%   Please, please: do it
% \vspace*{15pt}
%   \begin{itemize}
%   \item 
%   \end{itemize}
% \end{frame}


\begin{frame}
  \frametitle{Working through the notes before class}
%  Please, please: do it
  \begin{itemize}
  \item If you don't work on this on your own \ldots
    \begin{itemize}
    \item classes will go painfully slow
    \item we will all end up frustrated
    \end{itemize}
  \item (If some of you do work through it and some won't \ldots most
    likely those who don't will be left behind)
  \end{itemize}
\end{frame}


\begin{frame}
  \frametitle{In case I was not clear}
  Please, do work on this on your own before coming to class.

\vspace*{15pt}


What you get out of this class depends a lot on what you put into
    this class
\end{frame}



\begin{frame}
  \frametitle{Class notes}
  \begin{itemize}
  \item Available before class
  \item I you can, avoid printing them:
    \begin{itemize}
    \item all major operating systems have
    PDF viewers that allow you to incorporate annotations in the PDF
    itself and save them for posterity;
  \item notes might suffer minor changes up until right before class;
  \item ($\Rightarrow$ if you are reading these notes before class, 
    these notes themselves can suffer changes :-) )
    \end{itemize}
\end{itemize}
\end{frame}

\begin{frame}
  \frametitle{Grading}
  \begin{itemize}
  \item A final exam with practical cases
  \item (We need to see how wifi/network access works to decide if
    in-class or out-of-class)
  \end{itemize}
\end{frame}




\begin{frame}
  \frametitle{And where will classes be held?}
  \begin{itemize}
  \item Seminarios 4 and 5.
  \end{itemize}
\end{frame}



\begin{frame}
  \frametitle{Using my own computer}
  \begin{itemize}
  \item You are very welcome to bring your own laptop.
  \item Make sure you install the required software {\scriptsize  \Burl{http://ligarto.org/rdiaz/BM1-software-applied-stats.txt}}
  \item \textbf{Wi-fi (eduroam)}: \Burl{http://www.uam.es/servicios/ti/servicios/wifi/}
  \item \textbf{\red{Please make sure it works BEFORE coming to class}}
  \item Address questions/issues to CAU (we cannot help).
  \item Note: wifi/internet should not be really needed for classes, but it
    will be needed for the exam.
  \end{itemize}
\end{frame}



\begin{frame}
  \frametitle{}
  \begin{itemize}
  \item Who is planning on bringing his/her own laptop?
  \end{itemize}
\end{frame}


% 1.  Average mark on Self-evaluation quiz (the post-class quiz). A skipped quiz
% will count as a zero value in order to calculate the average. Please note that
% answers to Problem Quiz (the pre-class quiz) do NOT contribute to this average
% value.

% 2.  Grade on the Final exam. Taking this exam is not mandatory (but it is highly
% recommended).

% The final mark for bioinformatics will be calculated as:

% (0.6xSelf-Eval. average)+(0.4xExam)


\end{document}
